\PassOptionsToPackage{usenames,dvipsnames}{xcolor}
\documentclass[11pt]{article}
\usepackage{hyperref}
%\usepackage{fancyvrb}
%\usepackage{listings}
%\usepackage{graphicx} % Required for the inclusion of images
\usepackage{natbib}   % Required to change bibliography style to APA
%\usepackage{amsmath}  % Required for some math elements
%\usepackage{graphicx}
%\usepackage{listings}
%\usepackage{enumitem}
%\usepackage{makeidx}
%\usepackage{gensymb}
\usepackage[export]{adjustbox} %easy figure alignment

\usepackage[utf8]{inputenc}
\usepackage[greek,english]{babel}
\usepackage[LGR]{fontenc}
\usepackage{fontspec}
\usepackage{libertine}

\setlength\parindent{0pt} % Removes all indentation from paragraphs

\renewcommand{\labelenumi}{\alph{enumi}.} % Make numbering in the enumerate environment by letter rather than number (e.g. section 6)

\title{Χρήση Σταθμών HEPOS για τις Ερευνητικές και Ακαδημαϊκές Δραστηριότητες του Κέντρου Δορυφόρων Διονύσου, Ε.Μ.Π.} % Title
\author{Δημήτριος \textsc{Παραδείσης}\\ Καθηγητής Ε.Μ.Π.}
\date{27 Ιανουαρίου 2016} % Date for the report

\makeindex

\begin{document}

\maketitle % Insert the title, author and date

%\begin{center}
%\begin{tabular}{l r}
%First Revision: & December 4, 2015 \\
%Library Uri:    & \url{http://dionysos.survey.ntua.gr} \\
%Version:        & ΜΟΝΟ ΓΙΑ ΕΣΩΤΕΡΙΚΟ
%\end{tabular}
%\end{center}

%\begin{abstract}
%{\small 
%Με το παρόν κείμενο, το Κέντρο Δορυφόρων Διονύσου και το Εργαστήριο Ανώτερης Γεωδαισίας του Εθνικού Μετσόβιου Πολυτεχνείου, αιτούνται την διάθεση δεδομένων του δικτύου HEPOS της Εθνικό Κτηματολόγιο \& Χαρτογράφηση Α.Ε. (ΕΚΧΑ) για χρήση σε ερευνητικές και ακαδημαϊκές τους δραστηριότητες.
%}
%\end{abstract}
\clearpage

%\tableofcontents
\clearpage

%\section{Δραστηριότητα Εργαστηρίων}\label{drastiriotita}
Από την ίδρυσή τους έως και σήμερα, τα Κέντρο Δορυφόρων Διονύσου (ΚΔΔ) και Εργαστήριο Ανώτερης Γεωδαισίας (ΕΑΓ) του Εθνικού Μετσόβιου Πολυτεχνείου (ΕΜΠ), έχουν να επιδείξουν σημαντικό έργο στον τομέα της Δορυφορικής Γεωδαισίας, τόσο στον ερευνητικό όσο και στον ακαδημαϊκό τομέα. Το έργο αυτό υπογραμμίζεται από πλήθος συνεργασιών με ελληνικά αλλά και διεθνή ιδρύματα (ενδεικτικά \textit{Εθνικό Αστεροσκοπείο Αθηνών} (\texttt{NOA}), \textit{Massachusetts Institute of Technology} (\texttt{MIT}), \textit{University of Oxford}, \textit{Centre for Observation and Modelling of Earthquakes, Volcanoes and Tectonics} (\texttt{COMET}), κτλ) καθώς και δεκάδες δημοσιεύσεις σε επιστημονικά περιοδικά και συνέδρια (ενδεικτικά \cite{Billiris1991}, \cite{JGRB11137}, \cite{Cocard199939}, \cite{JGRB12107}, \cite{Reilinger201022}, \cite{JGRB16426}, \cite{Parks2012}, \cite{Ganas201362}).

Εδώ και δεκαετίες έχουν αναλάβει ή/και συμμετάσχει σε διάφορα ερευνητικά προγράμματα μεγάλου βεληνεκούς, αποκτώντας μεγάλη εμπειρία και τεχνογνωσία στην ανάλυση δορυφορικών γεωδαιτικών δεδομένων. Τα τελευταία χρόνια, τα εργαστήρια έχουν αναπτύξει μία ηλεκτρονική πλατφόρμα, αυτόματης ανάκτησης, αρχειοθέτησης, επεξεργασίας και δημοσίευσης δορυφορικών δεδομένων και αποτελεσμάτων (\cite{papanikegu}). Πιο αναλυτικά:

\begin{itemize}
\item Τα εργαστήρια ανακτούν δορυφορικά δεδομένα GNSS από πλήθος γεωδαιτικών σταθμών ανά την Ελλάδα $\cdot$ οι σταθμοί έχουν
εγκατασταθεί είτε από το ΕΜΠ ή από συνεργαζόμενους φορείς (π.χ. ΝΟΑ, COMET, Tree-Company). Ενδεικτικά, η λίστα των σταθμών είναι προσβάσιμη στην ιστοσελίδα \url{http://dionysos.survey.ntua.gr/dsoportal/_dataanalysis/BERN52PROC/stationlist.php}.

\item Τα δεδομένα αρχειοθετούνται και όταν αυτό είναι δυνατό (σύμφωνα με τις ενδείξεις του διαχειριστή), διανέμονται ελεύθερα μέσα από την διαδικτυακή πλατφόρμα \texttt{GSAC}\footnote{\url{https://www.unavco.org/projects/major-projects/gsac/gsac.html}}. Αξίζει να σημειωθεί, ότι η πλατφόρμα αυτή είναι το πλέον σύγχρονο μέσο διάχυσης γεωδαιτικών δεδομένων και προϊόντων και σχεδιάζεται να χρησιμοποιηθεί ως Ευρωπαϊκό πρώτυπο, μέσω του European Plate Observing System (EPOS \url{http://www.epos-eu.org/})\footnote{βλ. \url{https://www.unavco.org/projects/major-projects/gsac/lib/docs/GSAC_AGU_2013_Dec__Boler_et_al_poster.pdf}} $\cdot$ το ΕΜΠ ήταν από τα πρώτα Ευρωπαϊκά κέντρα που εγκατέστησε την πλατφόρμα, η οποία είναι ελεύθερα προσβάσιμη μέσω της ιστοσελίδας \url{http://dionysos.survey.ntua.gr/dsoportal/_datacenter/gsacrepos.html}.

\item Τα συλλεγόμενα δεδομένα υπόκεινται σε επεξεργασία για την εξαγωγή πλήθους παραμέτρων ενδιαφέροντος. Αυτές περιλαμβάνουν εκτιμήσεις συντεταγμένων, ατμοσφαιρικών παραμέτρων (ιονόσφαιρα, τροπόσφαιρα), τεκτονικών ταχυτήτων και μικρομετακινήσεων, βίαιων γεωφυσικών διεργασιων, κτλ. Τα εργαστήρια ακολουθούν τα πλέον σύγχρονα διεθνή πρότυπα ανάλυσης και διαθέτουν λογισμικά πακέτα υψήλης ακρίβειας (Bernese GNSS Software\footnote{http://www.bernese.unibe.ch/}, GAMIT/GLOBK\footnote{http://www-gpsg.mit.edu/~simon/gtgk/}, GIPSY/OASIS\footnote{https://gipsy-oasis.jpl.nasa.gov/}). Η ποιότητα των αποτελεσμάτων και των προϊόντων επιβεβαιώνεται από πλήθος σχετικών δημοσιεύσεων (ενδεικτικά \cite{anastacep}, \cite{GRL50066}) αλλά και από πρόσφατες συγκρίσεις που διεξήχθησαν από την υπηρεσία Analysis Combination Centre της EUREF.

\item Πολλά από τα αποτελέσματα και προϊόντα (της παραπάνω επεξεργασίας), διατίθενται ελεύθερα στην επιστημονική και ακαδημαϊκή κοινότητα, μέσω της ιστοσελίδας του Κέντρου Δορυφόρων Διονύσου (\url{http://dionysos.survey.ntua.gr/dsoportal/index.html}).
\end{itemize}

\vspace{0.5cm}
%\section{Αποτελέσματα}
Η δραστηριότητα των εργαστηρίων έχει βοηθήσει σημαντικά τόσο στην προώθηση και ανάπτυξη της Δορυφορικής Γεωδαισίας στη χώρα, όσο και στην μελέτη του πολύπλοκου τεκτονικού/γεωδυναμικού υποβάθρου της ευρύτερης Νοτιο-Ανατολικής Μεσογείου $\cdot$ μίας περιοχής που συνεχίζει να συγκεντρώνει παγκόσμιο επιστημονικό ενδιαφέρον λόγω της --σε μεγάλο βαθμό ακόμη ανεξιχνίαστης-- τεκτονικής ιδιομορφίας της.

Επίσης, έχει παραχθεί σημαντικό ακαδημαϊκό έργο, με πλήθος διπλωματικών εργασιών, αλλά και διδακτορικών διατριβών, που έχουν βασιστεί και υποστηρίξει την δραστηριότητα των εργαστηρίων.

\vspace{0.5cm}
%\section{Μέλλον}
Σκοπός των εργαστηρίων είναι να συνεχίσουν και να αναβαθμίσουν την ερευνητική και επιστημονική τους δραστηριότητα. Προς τον σκοπό αυτό,
\begin{enumerate}
\item Τα εργαστήρια παρακολουθούν και συμμετέχουν στις σύγχρονες εξελίξεις στον τομέα της Δορυφορικής Γεωδαισίας. Συγκεκριμένα, μεγάλο μέρος της πλατφόρμας που παρουσιάστηκε παραπάνω έχει ήδη αναβαθμιστεί και πρόκειται να τεθεί σε λειτουργία άμεσα. Τόσο ο εξοπλισμός, όσο και τα λογισμικά πακέτα έχουν ανανεωθεί και ελεγχθεί εκτενώς τους προηγούμενους μήνες (π.χ. \cite{mitseuref}) $\cdot$ τα εργαστήρια εξετάζουν και νέους τομείς ενδιαφέροντος (π.χ. real-time/near-real-time analysis).
\item Εξετάζονται δυνατότητες συμμετοχής σε διεθνή επιστημονικά προγράμματα και συνεργασία με ελληνικούς και διεθνείς φορείς.
\item Προετοιμάζεται η ενεργή συμμετοχή των εργαστηρίων στην EUREF, ώς κέντρο ανάλυσης.
\end{enumerate}

\vspace{0.5cm}
%\section{HEPOS}
Το δίκτυο HEPOS της ΕΚΧΑ αποτελεί ένα υψηλής ποιότητας δορυφορικό γεωδαιτικό δίκτυο. Ως τέτοιο, θα μπορούσε να βοηθήσει τα μέγιστα στις προσπάθειες των εργαστηρίων, τόσο στον ερευνητικό, όσο και στον ακαδημαϊκό τομέα. Πιο συγκεκριμένα:
\begin{itemize}
\item Η (αποδεδειγμένα) υψηλή ποιότητα του δικτύου, θα επιτρέψει την αποδοτική ανάλυση των δεδομένων με συνέπεια να επιτρέψει την εξαγωγή σημαντικών συμπερασμάτων για ποικίλες παραμέτρους ενδιαφέροντος. Η ποιότητα των δεδομένων αποτελεί καίριο και απαραίτητο παράγοντα για εφαρμογές ακριβείας, στις οποίες εξειδικεύονται τα εργαστήρια.
\item Το μεγάλο χρονικό διάστημα λειτουργίας του δικτύου, καθώς και η συνεχής συλλογή δεδομένων, είναι πολύ σημαντικός παράγοντας για την ποιοτική εκτίμηση παραμέτρων τεκτονικού ενδιαφέροντος (π.χ. ταχυτήτων). Δυστυχώς, τέτοιοι σταθμοί σπανίζουν στον Ελλαδικό χώρο και είναι ιδιαίτερα σημαντικοί.
\item Η χωρική κάλυψη (πυκνότητα) του δικτύου είναι ιδανική για μελέτη τόσο τοπικών φαινομένων, όσο και φαινομένων μεγαλύτερης κλίμακας. Τα δεδομένα που αναλύουν τα εργαστήρια έχουν σημαντικά χωρικά κενά που δυσχεραίνουν την εξαγωγή και συσχέτιση αποτελεσμάτων.
\item Η χρήση ποιοτικών και μακροχρόνιων παρατηρήσεων μπορεί να αναβαθμίσει τον ρόλο και την συμμετοχή των εργαστηρίων σε διεθνείς οργανισμούς κύρους (όπως η EUREF) με θετικά αποτελέσματα για όλη την Ελληνική γεωδαιτική κοινότητα.
\item Τα δεδομένα του δικτύου θα χρησιμοποιηθούν στην εκπαιδευτική διαδικασία $\cdot$ η επεξεργασία, διαχείρηση, ανάλυση σύγχρονων δορυφορικών δεδομένων είναι πλέον απαραίτητο εφόδιο για τον Τοπογράφο Μηχανικό. Η ποιότητα μάλιστα των δεδομένων, τα καθιστά ικανα να χρησιμοποιηθούν σε προπτυχιακό αλλά και μεταπτυχιακό επίπεδο.
\item Ο τρόπος συλλογής των δεδομένων (συνεχής, σε πραγματικό χρόνο), θα μπορούσε να επιτρέψει στα εργαστήρια να επεκτείνουν το φάσμα των ερενητικών τους δραστηριοτήτων. Πίο συγκεκριμένα, τα δεδομένα θα μπορούσαν να χρησιμοποιηθούν για την μελέτη φαινομένων σε σχεδόν πραγματικό χρόνο $\cdot$ το πεδίο αυτό αποτελεί έναν από τους πιο ραγδαία αναπτυσσόμενους τομείς της Δορυφορικής Γεωδαισίας.
\item Η ενσωμάτωση δεδομένων του HEPOS στο ήδη υπάρχον δίκτυο σταθμών που αναλύονται από τα εργαστήρια, θα μπορούσε να ενδυναμώση την ποιότητα και το εύρος των παραγόμενων αποτελεσμάτων και προϊόντων, με θετικές συνέπειες τόσο για την επιστημονική κοινότητα, όσο και για την ίδια την ΕΚΧΑ. Μέσω της ιστοσελίδας των εργαστηρίων, αλλά και της ιστοσελίδας/ιστοχώρου της ΕΚΧΑ, τα αποτελέσματα μπορούν να είναι ελεύθερα προσβάσιμα σε οποιονδήποτε ενδιαφερομένο.
\end{itemize}

Αποτελεί πεποίθηση των εργαστηρίων ότι η χρήση δεδομένων του δικτύου HEPOS μπορεί να βοηθήσει σημαντικά, να στηρίξει και να αναβαθμίσει την ερευνητική και ακαδημαϊκή τους δραστηριότητα.

\vspace{0.5cm}
%\section{Δεν το κανα ακόμη!}
%*BOOM**Με βάση τα παραπάνω, αιτούμαστε τη διάθεση από την ΕΚΧΑ δεδομένων για είκοσι (20) σταθμούς του δικτύου HEPOS, και για χρονικό διάστημα τεσσάρων (4) ετών, με σκοπό την παραγωγή ενός πλήρους κι ενημερωμένου πεδίου ταχυτήτων για τον ελληνικό χώρο, καθώς και την πύκνωση των τοπικών δικτύων της EUREF, μέσω του densification repro.
Έχοντας υπ 'όψην τα παραπάνω, τα εργαστήρια αιτούνται την παροχή από την ΕΚΧΑ δεδομένων του δικτύου HEPOS. Η παροχή δεδομένων για περίπου εικοσιένα (21) σταθμούς, κατανεμημένων σε κατάλληλες θέσεις ανά την Ελλάδα, για μία περίοδο τεσσάρων (4) χρόνων, θα μπορούσε να καλύψει μεγάλο μέρος των σημερινών αναγκών των εργαστηρίων.

\begin{figure}[h]
\begin{center}
\includegraphics[width=.8\textwidth]{../figures/hepos21.jpg}
%\caption{Προτεινόμενοι σταθμοί HEPOS.}
\end{center}
\end{figure}

Το είδος των δεδομένων που ζητούνται είναι αρχεία παρατήρησης (observation) RINEX (που ήδη χρησιμοποιεί και διανείμει η ΕΚΧΑ) με ρυθμό δειγματοληψίας 30 δευτερολέπτων. Η παροχή δεδομένων και ο τρόπος λήψης τους, μπορεί να γίνει με οποιονδήποτε τρόπο κριθεί κατάλληλος από την ΕΚΧΑ. Να σημειωθεί ότι η παροχή αυτή μπορεί να γίνεται με καθυστέρηση (δηλαδή τα δεδομένα να αποστέλλονται/λαμβάνονται αρκετές ημέρες ή/και εβδομάδες μετά τη συλλογή τους) αν αυτό κριθεί σκόπιμο, ώστε να μην εμποδίζεται αποιαδήποτε εμπορική δραστηριότητα της εταιρίας.

Τα εργαστήρια δεσμεύονται ότι:
\begin{itemize}
\item Δεν θα χρησιμοποιήσουν τα δεδομένα για εμπορικό σκοπό, και η χρήση τους θα αφορά αποκλειστικά τις ερευνητικές/εκπαιδευτικές δραστηριότητες των εργαστηρίων
\item Δεν θα διανείμουν δεδομένα της ΕΚΧΑ σε τρίτους,
\item Τα αποτελέσματα της ανάλυσης θα είναι άμεσα διαθέσιμα στην ΕΚΧΑ,
\item Οποιαδήποτε χρήση αποτελεσμάτων (για τους παραπάνω ερευνητικούς/εκπαιδευτικούς σκοπούς) από σταθμούς του HEPOS θα συνοδεύεται από τις κατάλληλες αναφορές (aknowledgment)
\end{itemize}

Δεδομένης της εμπειρίας των εργαστηρίων στην ανάλυση βίαιων τεκτονικών φαινομένων (όπως π.χ. σεισμοί), αλλά και της κρισιμότητας μελέτης τέτοιων διεργασιών σε διεθνές επίπεδο, τόσο για την επιστημονική κοινότητα όσο και για κρατικούς φορείς, θεωρούμε ότι θα πρέπει να υπάρχουν ανοιχτοί δίαυλοι επικοινωνίας μεταξύ της εταιρίας και των εργαστηρίων, ώστε σε περίπτωση ενός τέτοιου συμβάντος να μπορεί να υπάρξει έγκαιρη ανταλλαγή δεδομένων. Τα δεδομένα αυτά θα αφορούν μικρή χωρική έκταση, και περιορισμένο χρονικό διάστημα (ενδεικτικά λίγες ημέρες πριν και μετά την εκδήλωση του φαινομένου), με υψηλότερο ρυθμό δειγματοληψίας (π.χ. 1Hz). Η μελέτη τέτοιων φαινομένων μέσω δορυφορικών μεθόδων, αποτελεί καινοφανές επιστημονικό πεδίο, με τη χώρα μας να προβάλλει ώς το ιδανικό πεδίο μελέτης.

Ευελπιστούμε στην απαρχή μίας αγαστής και μακροχρόνιας συνεργασίας, της οποίας το παρόν είναι μόνο το πρώτο βήμα. Θεωρούμε δεδομένο ότι μία τέτοια συνεργασία θα βοηθήσει όχι μόνο τους άμεσα εμπλεκόμενους, αλλά και την Ελληνική και διεθνή επιστημονική κοινότητα.

\bibliographystyle{unsrt}
\bibliography{ekxa-dso}

%\addcontentsline{toc}{chapter}{Index}
%\printindex

\end{document}
