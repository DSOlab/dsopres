\PassOptionsToPackage{usenames,dvipsnames}{xcolor}
\documentclass{article}
\usepackage{hyperref}
%\usepackage{fancyvrb}
%\usepackage{listings}
%\usepackage{graphicx} % Required for the inclusion of images
%\usepackage{natbib}   % Required to change bibliography style to APA
%\usepackage{amsmath}  % Required for some math elements
%\usepackage{graphicx}
%\usepackage{listings}
%\usepackage{enumitem}
%\usepackage{makeidx}
%\usepackage{gensymb}

\usepackage{fontenc}
\usepackage[utf8]{inputenc}
\usepackage{fontspec}
\usepackage{libertine}
\usepackage[greek,english]{babel}

\setlength\parindent{0pt} % Removes all indentation from paragraphs

\renewcommand{\labelenumi}{\alph{enumi}.} % Make numbering in the enumerate environment by letter rather than number (e.g. section 6)

\title{Χρήση Σταθμών HEPOS για τις Ερευνητικές \& Ακαδημαικές Δραστηριότητες του Κέντρου Δορυφόρων Διονύσου, Ε.Μ.Π.} % Title
\author{Δημήτρης \textsc{Παραδείσης}, Καθ. Ε.Μ.Π.}
\date{\today} % Date for the report

\makeindex

\begin{document}

\maketitle % Insert the title, author and date

\begin{center}
\begin{tabular}{l r}
First Revision: & December 4, 2015 \\
Library Uri:    & \url{http://dionysos.survey.ntua.gr} \\
Version:        & v1.0-0
\end{tabular}
\end{center}

\begin{abstract}
{\small 
Με το παρών κείμενο, το Κέντρο Δορυφόρων Διονύσου και το Εργαστήριο Ανώτερης Γεωδαισίας του Εθνικού Μετσόβιου Πολυτεχνείου, αιτούνται την διάθεση δεδομένων του δικτύου HEPOS της Εθνικό Κτηματολόγιο \& Χαρτογράφηση Α.Ε. (ΕΚΧΑ) για χρήση σε ερευνητικές και ακαδημαϊκές τους δραστηριότητες.
}
\end{abstract}
\clearpage

\tableofcontents
\clearpage

\section{Δραστηριότητα Εργαστηρίων}
Από την ίδρυσή τους έως και σήμερα, τα Κέντρο Δορυφόρων Διονύσου (ΚΔΔ) και Εργαστήριο Ανώτερης Γεωδαισίας (ΕΑΓ) του Εθνικού Μετσόβιου Πολυτεχνείου (ΕΜΠ), έχουν να επιδείξουν σημαντικό έργο στον τομέα της Δορυφορικής Γεωδαισίας, τόσο στον ερευνητικό όσο και στον ακαδημαϊκό τομέα. Το έργο αυτό υπογραμμίζεται από πλήθος συνεργασιών με ελληνικά αλλά και διεθνή ιδρύματα (ενδεικτικά \textit{Εθνικό Αστεροσκοπείο Αθηνών} (\texttt{NOA}), \textit{Massachusetts Institute of Technology} (\texttt{MIT}), \textit{University of Oxford}, \textit{Centre for Observation and Modelling of Earthquakes, Volcanoes and Tectonics} (\texttt{COMET}), κτλ) καθώς και δεκάδες δημοσιεύσεις σε επιστημονικά περιοδικά και συνέδρια.




\addcontentsline{toc}{chapter}{Index}
%\printindex

\end{document}
