\PassOptionsToPackage{usenames,dvipsnames}{xcolor}
\documentclass{article}
\usepackage{tikz}
\usetikzlibrary{shapes,shapes.geometric, arrows}
\usepackage{hyperref}
\usepackage{fancyvrb}
\usepackage{listings}
\usepackage{graphicx} % Required for the inclusion of images
\usepackage{natbib}   % Required to change bibliography style to APA
\usepackage{amsmath}  % Required for some math elements
\usepackage{graphicx}
\usepackage{listings}
\usepackage{enumitem}
\usepackage{makeidx}
\usepackage{gensymb}
\usepackage[tikz]{bclogo} % Pretty formating of warning blocks

\setlength\parindent{0pt} % Removes all indentation from paragraphs

\renewcommand{\labelenumi}{\alph{enumi}.} % Make numbering in the enumerate environment by letter rather than number (e.g. section 6)

% warning box
\definecolor{warningbackground}{RGB}{252,226,158}
%\usepackage{floatflt}
\newcommand{\alertwarningbox}[1]{
    \centering
    \colorbox{warningbackground}{\parbox{400pt} {
            \vskip 10pt
            \begin{floatingfigure}[l]{50pt}
                \includegraphics[scale=.05]{img/danger.pdf}
            \end{floatingfigure}
            #1
            \vskip 10pt
        }
    }
}

\title{Routine GNSS Processing for EUREF Densification\\ Dionysos Satellite Observatory, NTUA} % Title
\author{Xanthos \textsc{Papanikolaou} \and Demitris \textsc{Anastasiou}}
\date{\today} % Date for the report

\makeindex

\begin{document}

\maketitle % Insert the title, author and date

\begin{center}
\begin{tabular}{l r}
First Revision: & December 4, 2015 \\
Library Uri:    & \url{http://dionysos.survey.ntua.gr} \\
Version:        & v1.0-0
\end{tabular}
\end{center}

\begin{abstract}
{\small 
This document describes the routine processing of GNSS data as performed 
by Dionysos Satellite Observatory (DSO), of National Technical University of 
Athens (NTUA) for EUREF Densification Project.
}
\end{abstract}
\clearpage

\tableofcontents
\clearpage

\section{General}
This document describes the routine processing of GNSS performed at DSO for
the EUREF Densification Project.
\clearpage

\section{Data and Products}
\subsection{RINEX files}
The required RINEX files are downloaded on a daily basis from the respective
data centers.
\subsection{Products}
We use the \texttt{final} products from Center for Orbit Determination in Europee
(CODE). The required products are downloaded from CODE's ftp site, namely
\url{ftp://ftp.unibe.ch/aiub/CODE/YYYY/}, where \texttt{YYYY} is the year.
\begin{center}
\begin{tabular}{l l l l}
Product & Type & Directory & Notes\\
\hline
Satellite orbits & sp3 & \texttt{YYYY/} & \\
Earth rotation parameters & erp & \texttt{YYYY/} & \\
Ionospheric corrections & ion & \texttt{YYYY/} & Bernese - specific\\
Code differential Bias & dcb & \texttt{YYYY/} & P1C1\\
VMF1 grid files & & \texttt{DELAY/GRID/VMFG/YYYY} \footnote{at \url{http://ggosatm.hg.tuwien.ac.at}.} & six-hour grid files\\
\hline
\end{tabular}
\end{center}
\subsection{A-priori Coordinates}
For \texttt{igs} stations we use their published coordinates in \texttt{IGb08}.
For EPN class A stations, we use their published coordinates, as extrapolated
from \url{ftp://ftp.epncb.oma.be/pub/station/coord/EPN/EPN\_A\_IGb08.SSC}.
\subsection{Ocean Loading}
Ocean loading corrections applied according to the model \texttt{FES2004}. For
EPN sites, displacements are extracted from \url{ftp://epncb.oma.be/pub/station/general/EPN\_FES2004.BLQ}.
\subsection{Receiver \& Satellite Antennae Calibration}
Corrections are extracted from the most recent EPN ANTEX file, i.e. 
\url{ftp://ftp.epncb.oma.be/pub/station/general/epn\_08.atx}.
\subsection{Excluded Stations}
We use the EUREF produced exlusion list (\url{ftp://ftp.epncb.oma.be/pub/station/general/excluded/excluded.wwww}, where 'wwww' is the gps week). All stations mentioned
therein are exluded from the processing.
\subsection{Metadata}
Station information are not extracted from the RINEX files; they are read from
the most recent version of the file \url{ftp://ftp.epncb.oma.be/pub/station/general/EUREF52.STA}.
\subsection{Reference Frame}
The network is aligned to the frame \texttt{IGb08} via the 'minimum-constraint-conditions'
approach. The sites used to that end are: \texttt{MISSING TEXT}.

Any site (out of the reference list) with offsets larger than 10mm in the North,
East or Up component, with respect to its published \texttt{IGb08} coordinates, is
not used as reference site.
\subsection{Elevation Angle \& Observation Weighting}
We use an elevation cut-off angle of 3\degree. Elevation dependent weighting 
of observations applied, according to the function \(1/cos^2(z)\).
\subsection{Tropospheric Refraction}
We use the VMF1 model; a priori ZHD extracted from the VMF grid files. ZWD estimated
for each station in intervals of 1 hour. Relative constraints of 5 m are applied.

Additionaly, horizontal delay gradient parameters are estimated for each station 
in intervals of 24 hours, according to Chen and Herring (1997); relative constraints of
5 m are applied.
\subsection{Ambiguity Fixing}
Ambiguity resolution is performed, using a baseline-length dependent scenario, 
as proposed by CODE, i.e.
\begin{description}
\item[Code-Based Widelane (WL)] 
For baselines shorter than 6000km, a Melbourne-Wuebbena wide-lane (after checking the residuals
of the code observations for outliers) and a narrow-lane ambiguity resolution is performed. This approach only used for GPS observations.
\item[Phase-Based Widelane (L5)] For baselines shorter than
200km the code-based wide-lane ambiguity resolution is replaced by a phase-only wide-lane
with a subsequent narrow-lane ambiguity resolution.
\item[Quasi-Ionosphere-Free (QIF)] The QIF-strategy is applied on the remaining real-valued
ambiguities for baselines shorter tahn 2000 km.
\item[Direct L1/L2] A direct L1/L2 ambiguity resolution is applied instead of
the above mentioned sequence of strategies on very short baselines, i.e. 20km.
\end{description}
\subsection{Satellite Systems}
When available, GLONASS observations are included in the processing (a pure
\clearpage

\addcontentsline{toc}{chapter}{Index}
\printindex

\end{document}
