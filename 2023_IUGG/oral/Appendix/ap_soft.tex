% \section{Συμπεράσματα}

\begin{frame}
  \frametitle{Ρουτίνες λογισμικού}
  \framesubtitle{}
\underline{\textbf{Αποθετήριο OS code}: \href{https://github.com/demanasta}{https://github.com/demanasta \faGithub}}
\vskip.2cm
\begin{footnotesize}
Το λογισμικό έχει αναπτυχθεί στα πλαίσια της διδακτορικής διατριβής και των ερευνητικών δραστηριοτήτων του Κέντρου Δορυφόρων Διονύσου και του Εργαστηρίου Ανώτερης Γεωδαισίας και διατίθεται υπό την άδεια GPL-v3.0 ως ελεύθερο λογισμικό/λογισμικό ανοιχτού κώδικα (ΕΛ/ΛΑΚ).
\vskip.3cm
\begin{tabular}{l p{9cm}}
\textbf{1. GeoToolbox:} & Ρουτίνες σε περιβάλλον Matlab για την ανάλυση των τεκτονικών ταχυτήτων και των υπολογισμό τανυστών ανηγμένης παραμόρφωσης (\href{http://demanasta.github.io/GeoToolbox/}{http://demanasta.github.io/GeoToolbox/}) \\
\textbf{2. gpsvel:} & Σχεδιασμός χαρτών τεκτονικών ταχυτήτων και τανυστών ανηγμένης παραμόρφωσης σε περιβάλλον GMT (\href{http://demanasta.github.io/gpsvel/}{http://demanasta.github.io/gpsvel/}) \\
\textbf{3. plot\_eq:} & Σχεδιασμός χαρτών καταλόγων σεισμών στην περιοχή της Ελλάδας (\href{http://demanasta.github.io/plot\_eq/}{http://demanasta.github.io/plot\_eq/}) \\
\textbf{4. GNSS\_nets:} & Σχεδιασμός χαρτών απεικόνισης δικτύων GNSS και αποτελεσμάτων της επεξεργασίας σε περιβάλλον GMT (\href{http://demanasta.github.io/GNSS\_nets/}{http://demanasta.github.io/GNSS\_nets/}) \\
\end{tabular}
\end{footnotesize}
\end{frame}
