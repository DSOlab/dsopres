% ************************ Pres Information & Meta-data ************************
% This file includes all available informations and meta-data for your presentation
% in four sections:
% 1. General & contact informations, for all styles.
% 2. PhD: Use this section with PhD style.
% 3. Pub: Use this section with publication style.
% 4. Lct: Usethis section with lecture style.
%
% Uncomment only one section of 2,3 or 4 each time.

%% -----------------------------------------------------------------------------
%% 1.General information... 
%% -----------------------------------------------------------------------------
% ************************ Pres Information & Meta-data ************************

%% Meta information
% \subject{Γεωδαισία} \keywords{{Γεωδαισία} {Τριγωνισμός} {Παραμόρφωση} {Ελλάδα}}

%% Contact e-informations
\urlhome{http://dionysos.survey.ntua.gr/}  %% homepage
\contmail{danastasiou@mail.ntua.gr}  %% contact mail
\urlin{https://www.linkedin.com/in/dganastasiou/}  %% linkedin url
\urlgh{https://github.com/demanasta}  %% github repository
%\urlgp{https://plus.google.com/u/0/+DemitrisAnastasiou}  %% Google+ 
\urltw{https://twitter.com/DemAnast}  %% Twitter

%% Add "thank you" text% 
\thankutext{Thank you for your attention!}

% %% -----------------------------------------------------------------------------
% %% 2.PhD section INFO
% %% -----------------------------------------------------------------------------
% %% ************************ Thesis Information & Meta-data **********************
% %% The title of the thesis
% \eltitle{ΠΡΟΤΥΠΟ ΠΑΡΟΥΣΙΑΣΗΣ ΣΕ ΠΑΡΙΒΑΛΛΟΝ \\ Beamer-\LaTeX / \XeLaTeX}
% 
% %% Subtitle (Optional)
% % \subtitle{Using the CUED template}
% 
% %% The full name of the author
% \authorname{ΔΗΜΗΤΡΙΟΣ Γ. ΑΝΑΣΤΑΣΙΟΥ}
% \authortitle{Διπλ. Αγρονόμος \& Τοπογράφος Μηχανικός Ε.Μ.Π}
% 
% %% Department (eg. Department of Engineering, Maths, Physics)
% \dept{ΣΧΟΛΗ ΑΓΡΟΝΟΜΩΝ \& ΤΟΠΟΓΡΑΦΩΝ ΜΗΧΑΝΙΚΩΝ}
% 
% %% Laboratory
% \lab{ΚΕΝΤΡΟ ΔΟΡΥΦΟΡΩΝ ΔΙΟΝΥΣΟΥ}
% 
% %% University and Crest
% \university{ΕΘΝΙΚΟ ΜΕΤΣΟΒΙΟ ΠΟΛΥΤΕΧΝΕΙΟ}
% 
% % Crest minimum should be 30mm.
% \crestleft{\includegraphics[width=\textwidth,draft=false]{Figs/ntua.png}}
% \crestright{\includegraphics[width=0.85\textwidth,draft=false]{Figs/DSOtrans.png}}
% 
% %% Full title of the Degree
% \degreetitle{ΔΙΔΑΚΤΟΡΙΚΗ ΔΙΑΤΡΙΒΗ}
% 
% % Supervisor
% \supervisor{......O/E.........\\ ....Θέση..........}
% 
% %% College affiliation (optional)
% \city{ΑΘΗΝΑ}
% 
% %% Submission date
% % Default is set as {\monthname[\the\month]\space\the\year}
% % \degreedate{\today} 
% \degreedate{5 Ιουλίου 2017}



%% -----------------------------------------------------------------------------
%% 3.Publication's section INFO
%% -----------------------------------------------------------------------------
%% The title of the thesis
\prestitle{Velocity and strain field estimation\\ using permanent GNSS stations\\ in the region of the EnCeladus Hellenic Supersite}

%% The team prepare this presentation
\presteam{
\underline{Dimitrios Anastasiou$^{1}$},
Xanthos Papanikolaou$^{1}$,
Maria Tsakiri$^{1}$,
Spyros Lalechos$^{2}$}

%% Organizations of the team
\presorgn{$^{1}$Dionysos Satellite Observatory, School of Rural Surveying and Geoinformatics Engineering \\ National Technical University of Athens\\
$^{2}$Earthquake Planning and Protection Organization
}
%Contact informations
\presweb{dionysos.survey.ntua.gr}  % webpage
\presmail{danastasiou@mail.ntua.gr}  % contact mail

%% Conference details, Select  text or logo type. If you define both only logo will
%% be print
% \confname{12\textsuperscript{th} HSTAM International Congress on Mechanics}
% \confdetail{Thessaloniki, Greece, 22 - 25 September 2019}

%% OR conf logo....
\conflogo{\includegraphics[width=.85\textwidth,draft=false]{Figs/IUGG2023_logo.png}}


%% -----------------------------------------------------------------------------
%% 4.Course section INFO
%% -----------------------------------------------------------------------------
% 
% %%% Department (eg. Department of Engineering, Maths, Physics)
% \dept{ΣΧΟΛΗ ΑΓΡΟΝΟΜΩΝ \& ΤΟΠΟΓΡΑΦΩΝ ΜΗΧΑΝΙΚΩΝ}
% 
% %% Laboratory
% \lab{ΚΕΝΤΡΟ ΔΟΡΥΦΟΡΩΝ ΔΙΟΝΥΣΟΥ}
% 
% %% University and Crest
% \university{ΕΘΝΙΚΟ ΜΕΤΣΟΒΙΟ ΠΟΛΥΤΕΧΝΕΙΟ}
% 
% % Crest minimum should be 30mm.
% \crestleft{\includegraphics[width=\textwidth,draft=false]{Figs/ntua.png}}
% \crestright{\includegraphics[width=0.85\textwidth,draft=false]{Figs/DSOtrans.png}}
% 
% 
% %% The full name of the author
% \authorname{ΔΗΜΗΤΡΙΟΣ Γ. ΑΝΑΣΤΑΣΙΟΥ}
% \authortitle{Δρ. Αγρ. \& Τοπ. Μηχ. Ε.Μ.Π}
% 
% %% Lecture title
% \coursetitle{Τίτλος του Μαθήματος}
% \courseinfo{5ο Εξάμηνο}
% \lcttitle{Δημιουργία παρουσιάσεων σε περιβάλλον {\LaTeX}}
% 
% 
% %%% College affiliation (optional)
% \city{ΑΘΗΝΑ}
% 
% %% Submission date
% % Default is set as {\monthname[\the\month]\space\the\year}
% % \degreedate{\today} 
% \coursedate{01 Οκτωβρίου 2017}










