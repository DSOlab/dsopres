%%%%%%%%%%%%%%%%%%%%%%%%%%%%%%%%%%%%%%%%%
% Professional Formal Letter
% LaTeX Template
% Version 1.0 (28/12/13)
%
% This template has been downloaded from:
% http://www.LaTeXTemplates.com
%
% Original author:
% Brian Moses (http://www.ms.uky.edu/~math/Resources/Templates/LaTeX/)
% with extensive modifications by Vel (vel@latextemplates.com)
%
% License:
% CC BY-NC-SA 3.0 (http://creativecommons.org/licenses/by-nc-sa/3.0/)
%
%%%%%%%%%%%%%%%%%%%%%%%%%%%%%%%%%%%%%%%%%

%----------------------------------------------------------------------------------------
%	PACKAGES AND OTHER DOCUMENT CONFIGURATIONS
%----------------------------------------------------------------------------------------

\documentclass[11pt,a4paper]{letter} % Specify the font size (10pt, 11pt and 12pt) and paper size (letterpaper, a4paper, etc)

\usepackage{graphicx} % Required for including pictures
\usepackage{microtype} % Improves typography
\usepackage{gfsdidot} % Use the GFS Didot font: http://www.tug.dk/FontCatalogue/gfsdidot/
\usepackage[LGR]{fontenc} % Required for accented characters
\usepackage{xgreek}
\usepackage{fontspec}
\usepackage{libertine}

% Create a new command for the horizontal rule in the document which allows thickness specification
\makeatletter
\def\vhrulefill#1{\leavevmode\leaders\hrule\@height#1\hfill \kern\z@}
\makeatother

%----------------------------------------------------------------------------------------
%	DOCUMENT MARGINS
%----------------------------------------------------------------------------------------

\textwidth 6.75in
\textheight 13.25in
\oddsidemargin -.25in
\evensidemargin -.25in
\topmargin -2.6in
\longindentation 0.50\textwidth
\parindent 0.4in

%----------------------------------------------------------------------------------------
%	SENDER INFORMATION
%----------------------------------------------------------------------------------------

\def\Who{Δημήτριος Αναστασίου} % Your name
\def\What{Επίκουρος Καθηγητής ΕΜΠ} % Your title
\def\Where{} % Your department/institution
\def\Address{Ηρώων Πολυτεχνείου 9, } % Your address
\def\CityZip{ΤΚ:15780, Ζωγράφου, } % Your city, zip code, country, etc
\def\Email{E-mail: dso@survey.ntua.gr} % Your email address
\def\TEL{Τηλ: 210 772 2735} % Your phone number
\def\URL{URL: http://dionysos.survey.ntua.gr} % Your URL

%----------------------------------------------------------------------------------------
%	HEADER AND FROM ADDRESS STRUCTURE
%----------------------------------------------------------------------------------------

\address{
\includegraphics[width=1.8cm]{../../figures/DSOtrans.png} % Include the logo of your institution
\hspace{5.8in} % Position of the institution logo, increase to move left, decrease to move right
\vskip -1.05in~\\ % Position of the text in relation to the institution logo, increase to move down, decrease to move up
\large\hspace{1in}\textbf{ΕΘΝΙΚΟ ΜΕΤΣΟΒΙΟ ΠΟΛΥΤΕΧΝΕΙΟ} \hfill ~\\[0.05in] % First line of institution name, adjust hspace if your logo is wide
\hspace{1in}\textbf{ΣΧΟΛΗ ΑΓΡΟΝΟΜΩΝ \& ΤΟΠΟΓΡΑΦΩΝ ΜΗΧΑΝΙΚΩΝ - ΜΗΧΑΝΙΚΩΝ} \hfill ~\\[0.05in]  % Second line of institution name, adjust hspace if your logo is wide
\hspace{1in}\textbf{ΓΕΩΠΛΗΡΟΦΟΡΙΚΗΣ} \hfill ~\\[0.05in]
\hspace{1in}\textbf{ΚΕΝΤΡΟ ΔΟΡΥΦΟΡΩΝ ΔΙΟΝΥΣΟΥ} \hfill \normalsize % Second line of institution name, adjust hspace if your logo is wide
%\makebox[0ex][r]{\bf \Who \What }\hspace{0.08in} % Print your name and title with a little whitespace to the right
~\\[-0.11in] % Reduce the whitespace above the horizontal rule
\hspace{1in}\vhrulefill{1pt} \\ % Horizontal rule, adjust hspace if your logo is wide and \vhrulefill for the thickness of the rule
\hspace{\fill}\parbox[t]{4.85in}{ % Create a box for your details underneath the horizontal rule on the right
\footnotesize % Use a smaller font size for the details
%\Who \\ \em % Your name, all text after this will be italicized
%\Where\\ % Your department
\Address % Your address
\CityZip % Your city and zip code
\TEL\\ % Your phone number
\hfill\Email -- % Your email addres
\URL % Your URL
}
\hspace{-1.4in} % Horizontal position of this block, increase to move left, decrease to move right
\vspace{-1in} % Move the letter content up for a more compact look
}

%----------------------------------------------------------------------------------------
%	TO ADDRESS STRUCTURE
%----------------------------------------------------------------------------------------

\def\opening#1{\thispagestyle{empty}
{\centering\fromaddress \vspace{1.3in} \\ % Print the header and from address here, add whitespace to move date down
\hspace*{\longindentation}\hfill\today\hspace*{\fill}\par} % Print today's date, remove \today to not display it
{\raggedright  \toaddress \par} % Print the to name and address
\vspace{0.2in} % White space after the to address
\noindent #1 % Print the opening line
% Uncomment the 4 lines below to print a footnote with custom text
%\def\thefootnote{}
%\def\footnoterule{\hrule}
%\footnotetext{\hspace*{\fill}{\footnotesize\em Footnote text}}
%\def\thefootnote{\arabic{footnote}}
}

%----------------------------------------------------------------------------------------
%	SIGNATURE STRUCTURE
%----------------------------------------------------------------------------------------

\signature{\Who \\ \What} % The signature is a combination of your name and title

\long\def\closing#1{
\vspace{0.4in} % Some whitespace after the letter content and before the signature
\noindent % Stop paragraph indentation
\hspace*{\longindentation} % Move the signature right
\parbox{\indentedwidth}{\raggedright
#1 % Print the signature text
\vskip 0.65in % Whitespace between the signature text and your name
\fromsig}} % Print your name and title

%----------------------------------------------------------------------------------------

\begin{document}

%----------------------------------------------------------------------------------------
%	TO ADDRESS
%----------------------------------------------------------------------------------------

\begin{letter}


%----------------------------------------------------------------------------------------
%	LETTER CONTENT
%----------------------------------------------------------------------------------------

\opening{ΠΡΟΣ METRICA Α.Ε.,}

Την Δευτέρα 06-02-2023 έλαβαν χώρα δύο μεγάλοι σεισμοί στα σύνορα των χωρών Τουρκίας και Συρίας. Σε αναλύσεις δεδομένων 1Hz του σταθμού DYNG00GRC που είναι εγκατεστημένος στο Διόνυσο Αττικής παρατηρήθηκε ελαστική μετακίνηση. 
\begin{itemize}\setlength\itemsep{.5em}
  \item Με στόχο την μελέτη της επίδρασης των δύο μεγάλων σεισμών στην περιοχή της Ελλάδας, αιτούμαστε την διάθεση δεδομένων με συχνότητα καταγραφής 1Hz (ή μεγαλύτερης εάν υπάρχουν) των σταθμών του δικτύου HxGN SmartNET (επισυνάπτεται λίστα) για τις ημέρες 5, 6 και 7 Φεβρουαρίου 2023 (036-037-038). Η επεξεργασία των δεδομένων θα πραγματοποιηθεί με τη μέθοδο PPP για την εξέταση πιθανής ελαστικής παραμόρφωσης και την επίδραση των σεισμικών κυμάτων στην περιοχή της Ελλάδας. Η επεξεργασία θα γίνει μεμονομένα χωρίς ανάμειξη άλλων δικτύων εκτός των διεθνών σταθμών και προϊόντων της IGS.
  \item Η επεξεργασία των δεδομένων θα ολοκληρωθεί μέχρι το τέλος Απρίλη με στόχo και την δημοσίευση των αποτελεσμάτων στο ερχόμενο συνέδριο της IUGG 2023
  \item Στην ανάλυση των δεδομένων και αποτελεσμάτων θα συμμετέχουν, ο υπογράφων (Δημήτριος Αναστασίου), η Δρ. Μαρία Τσακίρη, καθηγήτρια ΕΜΠ, ο κ. Ιορδάνης Γαλάνης, μέλος ΕΤΕΠ ΕΜΠ και οι ΥΔ Βασιλική Κρέη και Ξάνθος Παπανικολάου.
  \item Όλα τα στοιχεία που θα διατεθούν από την εταιρεία METRICA A.E. και το δίκτυο Σταθμών Αναφοράς HxGN SmartNet, θα χρησιμοποιηθούν αυστηρά για ερευνητικούς ή εκπαιδευτικούς σκοπούς και όχι για εμπορικούς ή άλλους σκοπούς. Πρόσβαση στα δεδομένα αυτά έχουν μόνο όσοι συμμετέχουν στο πρόγραμμα και αποκλείεται η χρήση τους από τρίτους που δεν αναφέρονται στην παρούσα επιστολή. Η METRICA A.E. και το δίκτυο HxGN SmartNet δεν φέρουν καμία ευθύνη για τα αποτελέσματα και τα πορίσματα της ερευνητικής εργασίας που θα προκύψουν από την χρήση των δεδομένων. Σε κάθε ανακοίνωση ή εργασία που θα παραχθεί με τα δεδομένα του HxGN SmartNet θα γίνεται σαφής αναφορά τόσο στην METRICA A.E. όσο και στο δίκτυο HxGN SmartNet.
\end{itemize}
Λϊστα προτεινόμενων σταθμών: \\
RODO - KRP1 - KALY - SAMO - CHIO - LESV - IKAR - AST1 - MYKN - NAXO - SANT - ANDR - SYR1 - MILO - SKYR - KYMI - HALK - MET0 - ANAV - ISTI - LAMA - THIV - KORI -  SITI - IERA - ARKL - MOI1 - SPAK - KISM - ANKY - NEAB - PTKG - PYLO
\vskip-1cm
\closing{Με εκτίμηση,}

%----------------------------------------------------------------------------------------

\end{letter}
\end{document}
