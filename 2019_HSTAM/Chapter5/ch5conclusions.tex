\section{Conclusions}
 
% \graphicspath{Figs/}

\begin{frame}
 \frametitle{Conclusions}
 \framesubtitle{StrainTool open-source software v1.0}
 \label{ch5:concl}
  
  \begin{itemize}
    \item We propose a new, open-source tool to estimate strain in geodesy and geodynamics, the \texttt{STRAINTOOL}
    \item Free, flexible and cross-platform-compatibility.
    \item Use different algorithms to estimate strain tensor parameters.
    \item We validated our calculations using two open-source algorithms recommended by EPOS-IP, namely the VISR and STIB as well as the SSPX software suite.
    \item Οur results reproduce the gross features of tectonic deformation in both Italy and Greece, such as NE-SW extension across the Apennines and N-S extension in Central Greece.
    \item It is anticipated that the significant increase of GNSS data amount associated with the operational phase of EPOS in the forthcoming years will be of great value to perform an unprecedented, reliable strain rate computation over the Eurasian plate.
  \end{itemize}
\end{frame}
\note{}


\begin{frame}
 \frametitle{Conclusions}
 \framesubtitle{Tectonic strain in Eurasia}
 \label{ch5:concl}
  \begin{itemize}
    \item
    \item NE-SW extension across the Apennines.
    \item N-S extension in Central Greece.
    \item NE-SW compression acrossthe area of Albania's shoreline.
  \end{itemize}

\end{frame}
\note{}

%\begin{frame}
%  \frametitle{}
%  \framesubtitle{}
%  \label{ch5:concl}

%\end{frame}
%\note{}
