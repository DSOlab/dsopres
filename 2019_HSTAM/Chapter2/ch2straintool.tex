\section{Open Source Software \textbf{StrainTool v1.0}}
 
% \graphicspath{Figs/}

\begin{frame}
  \frametitle{Open Source Software \textbf{StrainTool v1.0}}
  \framesubtitle{}
  \label{ch2:straintool}
  
  StrainTool has three basic components:
  \begin{itemize}
    \item \textbf{pystrain:} A python pachage.
    \item \textbf{StrainTensor.py:} the main executable.
    \item A list of shell scripts to plot results from StrainTensor.py
  \end{itemize}
  \textcolor{red}{TODO: structure design}
  
\end{frame}
\note{}


\begin{frame}
  \frametitle{Python Package \texttt{pystrain}}
  \framesubtitle{}
  \label{ch2:}
  
  \texttt{pystrain} the core part of the project.
  
  Python functions and classes, enable computation of strain tensor.
  
  The package includes:
  \begin{itemize}
    \item \texttt{iotools}: input/output classes to parse ASCII files.
    \item \texttt{geodesy}: functions for basic geodetic calculations.
    \item \texttt{grid.py}: a simple grid generator
    \item \texttt{strain.py}: main class and necessary functions for estimation of strain  tensor parameters
  \end{itemize}
  
  

\end{frame}
\note{}


\begin{frame}
 \frametitle{Estimate strain tensor parameters}
 \framesubtitle{}
 \label{ch2:}

\end{frame}
\note{}

\begin{frame}
 \frametitle{Shen Algorithm}
 \framesubtitle{}
 \label{ch2:}

\end{frame}
\note{}


\begin{frame}
 \frametitle{Shen Algorithm}
 \framesubtitle{Distance-dependent weighting}
 \label{ch2:}

\end{frame}
\note{}

\begin{frame}
 \frametitle{Shen Algorithm}
 \framesubtitle{Optimal smoothin parameter D}
 \label{ch2:}

\end{frame}
\note{}

\begin{frame}
 \frametitle{Shen Algorithm}
 \framesubtitle{Spatial weights}
 \label{ch2:}

\end{frame}
\note{}

\begin{frame}
 \frametitle{Veis Algorithm}
 \framesubtitle{}
 \label{ch2:}

\end{frame}
\note{}



%\begin{frame}
%   \frametitle{}
%   \framesubtitle{}
%   \label{ch2:}

%\end{frame}
%\note{}
