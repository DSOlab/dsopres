\section{Open Source Software \textbf{StrainTool v1.0}}
 
% \graphicspath{Figs/}

\begin{frame}
  \frametitle{Open Source Software \textbf{StrainTool v1.0}}
  \framesubtitle{}
  \label{ch2:straintool}
  
  StrainTool has three basic components:
  \begin{itemize}
    \item \textbf{pystrain:} A python pachage.
    \item \textbf{StrainTensor.py:} the main executable.
    \item A list of shell scripts to plot results from StrainTensor.py
  \end{itemize}
  \textcolor{red}{TODO: structure design}
  
\end{frame}
\note{}


\begin{frame}
  \frametitle{Python Package \texttt{pystrain}}
  \framesubtitle{}
  \label{ch2:}
  
  \texttt{pystrain} the core part of the project.
  
  Python functions and classes, enable computation of strain tensor.
  
  The package includes:
  \begin{itemize}
    \item \texttt{iotools}: input/output classes to parse ASCII files.
    \item \texttt{geodesy}: functions for basic geodetic calculations.
    \item \texttt{grid.py}: a simple grid generator
    \item \texttt{strain.py}: main class and necessary functions for estimation of strain  tensor parameters
  \end{itemize}
  
  

\end{frame}
\note{}


\begin{frame}
 \frametitle{Estimate strain tensor parameters}
 \framesubtitle{}
 \label{ch2:}
 
 Strain tensor parameters aestimated (or calculated) by solving for the system:
 
 \[
  \begin{bmatrix}
    V_{x,S_1} \\ 
    V_{y,S_1} \\ 
    \cdots \\ 
    V_{x,S_n} \\ 
    V_{y,S_n} \\ 
  \end{bmatrix}
  =
  \begin{bmatrix}
    1 & 0 & \Delta_{y_1}  & \Delta_{x_1} & \Delta_{y_1} & 0 \\
    0 & 1 & -\Delta_{x_1} &  0           & \Delta_{x_1} & \Delta_{y_1} \\
    \cdots & \cdots & \cdots & \cdots & \cdots & \cdots \\
    1 & 0 & \Delta_{y_n}  & \Delta_{x_n} & \Delta_{y_n} & 0 \\
    0 & 1 & -\Delta_{x_n} &  0           & \Delta_{x_n} & \Delta_{y_n} \\
  \end{bmatrix}
  \begin{bmatrix}
    U_{x} \\ 
    U_{y} \\ 
    \omega \\ 
    \tau_{x} \\ 
    \tau_{xy} \\ 
    \tau_{y} \\ 
  \end{bmatrix}
  \]
  
  $\Delta_{x_i}, \Delta_{y_i}$ are the displacement components between station i and the point.
  
  A minimum of three stations is required to compute the parameters.
  

\end{frame}
\note{}

\begin{frame}
 \frametitle{Estimate strain tensor parameters}
 \framesubtitle{}
 \label{ch2:}
 
 Assuming that there is a variance information for the station velocities (and a Gaussian distribution), we can add the covariance matrix C  of the velocity data in the system. In the simplest case, C is a diagonal matrix, with the velocity component standard deviations as its elements.
 
 \[
 C = \sigma_{0}^{2} 
 \begin{bmatrix}
 (\frac{1}{\sigma_{V_{x_{1}S_{1}}}})^{2} & 0 & 0  & ... & 0\\
 0 & (\frac{1}{\sigma_{V_{y_{1}S_{1}}}})^{2} & 0  & ... & 0\\
 0 & 0 & ({\frac{1}{\sigma_{V_{x_{2}S_{2}}}}})^{2} & ... & 0\\
 ... &  ... & ... & ... & ...\\
 0 & 0 & 0 & ... & (\frac{1}{\sigma_{V_{y_{i}S_{i}}}})^{2}\\
 \end{bmatrix}
 \]

\end{frame}
\note{}


\begin{frame}
 \frametitle{Shen Algorithm}
 \framesubtitle{}
 \label{ch2:}

\end{frame}
\note{}


\begin{frame}
 \frametitle{Shen Algorithm}
 \framesubtitle{Distance-dependent weighting}
 \label{ch2:}

\end{frame}
\note{}

\begin{frame}
 \frametitle{Shen Algorithm}
 \framesubtitle{Optimal smoothin parameter D}
 \label{ch2:}

\end{frame}
\note{}

\begin{frame}
 \frametitle{Shen Algorithm}
 \framesubtitle{Spatial weights}
 \label{ch2:}

\end{frame}
\note{}

\begin{frame}
 \frametitle{Veis Algorithm}
 \framesubtitle{}
 \label{ch2:}

\end{frame}
\note{}



%\begin{frame}
%   \frametitle{}
%   \framesubtitle{}
%   \label{ch2:}

%\end{frame}
%\note{}
